\documentclass{article}

% Language setting
% Replace `english' with e.g. `spanish' to change the document language
\usepackage[spanish]{babel}

% Set page size and margins
% Replace `letterpaper' with `a4paper' for UK/EU standard size
\usepackage[letterpaper,top=2cm,bottom=2cm,left=3cm,right=3cm,marginparwidth=1.75cm]{geometry}

% Useful packages
\usepackage{amsmath}
\usepackage{graphicx}
\usepackage{enumitem}
\usepackage{comment}
\usepackage{wrapfig}
\usepackage{amssymb}
\usepackage[colorlinks=true, allcolors=blue]{hyperref}

\title{Bases de Datos I Tema 4. Lenguajes formales de consulta}
\author{Martín González Dios 
\href{https://github.com/martindios}{\includegraphics[height=0.5cm]{github.png}}}

\begin{document}
\maketitle

\section{Lenguajes de consulta relacional}
(\textbf{Lenguaje relacionalmente completo}: puede producir todas las relaciones)
Existen 2 tipos:
\begin{itemize}
    \item \textbf{Lenguajes procedimentales}: el usuario indica al sistema que lleve a cabo una serie de operaciones en la BD para calcular el resultado deseado.

    \item \textbf{Lenguajes no procedimentales}: el usuario describe la información deseada sin establecer un procedimiento concreto para obtener esa información. 
\end{itemize}

\section{Operaciones relacionales}
Existen 5 operaciones básicas en el Álgebra Relacional: \textbf{selección, proyección, producto cartesiano, unión de conjuntos y diferencia de conjuntos}. \\
Como todas estas operaciones tienen cómo salida una relación, las operaciones pueden \textbf{concatenarse}.

\section{Álgebra relacional}
\subsection{Selección}
Selecciona tuplas que satisfacen un predicado dado, se usa $\sigma$ (sigma) para denotar la selección. \\
Ejemplo: 
$$\sigma_{\text{nombre\_dept} = \text{<<Física>>}}(\text{Profesor})$$
Selecciona las tuplas de la relación Profesor en las que el profesor pertenece al departamento de Física. 

$$\sigma_{\text{nombre\_dept} = \text{<<Física>> AND(capirote) sueldo $>$ 90000}}(\text{Profesor})$$

\subsection{Proyección}
Operación unaria que devuelve su relación de argumentos, excluyendo algunos atributos. Se denomina por la letra pi ($\pi$). 

$$\pi_{\text{ID, nombre, sueldo}}\text{(Profesor)}$$
Relación de todos los ID, nombre y sueldo de los profesores.

\newpage

\subsection{Producto cartesiano}
Permite combinar información de dos relaciones cualesquiera. Se denomina con aspa (X). \textbf{La cantidad de atributos se suma, la cantidad de tuplas se multiplica}. \\
Es necesario filtrar el producto cartesiano, ya que se generan tuplas sin sentido.

$$\sigma_{\text{DNI, Estudiantes}= \text{estudante, Teléfonos}}\text{(Estudiantes X Teléfonos)}$$

En el caso de que los atributos se llamen igual (DNI y estudiante) puedo hacer una reunión natural. Estudiantes $\Join$ Telefonos es lo mismo que $$\sigma_{\text{DNIest = DNItelf}}\text{(Estudiantes X Teléfonos)}$$

La \textbf{reunión zeta} puede ser vista como una operación que toma dos relaciones (o tablas) y produce una nueva relación que contiene todas las combinaciones de filas de las dos relaciones que satisfacen la condición de unión especificada.

\subsection{Unión, intersección y diferencia de conjuntos}
\begin{itemize}
    \item \textbf{Unión}: se indica mediante $\cup$. Se sumas todas las tuplas en una misma tabla.

    \item \textbf{Intersección}: se indica mediante $\cap$. Se interseccionan las tuplas que existan en las 2 tablas.

    \item \textbf{Diferencia de conjuntos}: se indica mediante -. Permite encontrar tuplas que están en una relación, pero no en la otra. 
\end{itemize}

Estas 3 operaciones requieren que las \textbf{tablas} sean \textbf{compatibles}: 
\begin{itemize}
    \item Mismo número de atributos.
    \item Compatibilidad sintáctica (los atributos tienen que tener el mismo tipo de dato).
\end{itemize}

\section{Agregación}
Toman una colección de valores y devuelven como resultado un único valor. Se representa como $G$. 
$$G_{\text{contar}}\text{(Matrículas) = cuantas tuplas tiene la tabla Matrículas}$$

$$G_{\text{sumar(sueldo)}}\text{(Profesor) = suma de los sueldos de la tabla Profesor}$$

\section{Cálculo relacional de tuplas}
Es un lenguaje de consulta no procedimental para manipular el MR. \\
\{t$|$P(t)\} \\

$$\text{\{t}|\text{t}\in \text{Matrículas AND t(calificación) = No apto\}}$$
Las tuplas que pertenecen a Matrículas y el atributo calificación sea No Apto. \\

$$\text{\{t }|\text{ t}\in \text{Estudiantes } \exists \text{s} \in \text{ Matrículas } | \text{ t(dni) = s(estudiante) AND s(calificación) = Apto\}}$$

\newpage

\section{Cáclulo relacional de dominios}
Es otro lenguaje de consulta no procedimental para manipular el MR. \\
\{$<$dni, nombre, apellido1, apellido2, fechaNacimiento$>$ $|$ P(dni, nombre, ...)\} \\

\{$<$estudiante, materia, año, mes, calificación$>$ $|$ $<$e, m, a, ms, c$>$ $\in$ Matrículas AND c = No apto\} \\

\{$<$dni, n, ap1, ap2$>$ $|$ $<$dni, n, ap1, ap2$>$ $\in$ Estudiantes AND $<$e, c$>$ $\in$ Matrículas AND e = dni AND c = Apto\}

\begin{comment}
\begin{figure}[h]
    \centering
    \includegraphics[width=0.5\textwidth]{1.png}
    \caption{}
\end{figure}
\end{comment}

\begin{comment}
\begin{wrapfigure}[]{r}{0.5\linewidth}
    \centering
    \includegraphics[width=\linewidth]{8.png}
    \caption{}
\end{wrapfigure}
\end{comment}

\end{document}