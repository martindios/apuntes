\documentclass{article}

% Language setting
% Replace `english' with e.g. `spanish' to change the document language
\usepackage[spanish]{babel}

% Set page size and margins
% Replace `letterpaper' with `a4paper' for UK/EU standard size
\usepackage[letterpaper,top=2cm,bottom=2cm,left=3cm,right=3cm,marginparwidth=1.75cm]{geometry}

% Useful packages
\usepackage{amsmath}
\usepackage{graphicx}
\usepackage{enumitem}
\usepackage{comment}
\usepackage{wrapfig}
\usepackage[colorlinks=true, allcolors=blue]{hyperref}

\title{Bases de Datos I Tema 5. SQL: tablas y vistas}
\author{Martín González Dios 
\href{https://github.com/martindios}{\includegraphics[height=0.5cm]{github.png}}}

\begin{document}
\maketitle
\textbf{SQL (Structured Query Language)}.
\begin{itemize}
    \item \textbf{LDD}: \textbf{Lenguaje de Definición de Datos}; definición, modificación y borrado de tablas y vistas.

    \item \textbf{LMD}: \textbf{Lenguaje de Manipulado de Datos}; inserción, modificación y borrado de tuplas en las tablas.
\end{itemize}

\section{Definición de datos en SQL}
Tipos de datos:
\begin{itemize}
    \item \textbf{char(n)}: cadena de caracteres de longitud fija.
    
    \item \textbf{varchar(n)}: cadena de caracteres de longitud variable, con una longitud máxima $n$ especificada por el usuario.

    \item \textbf{int} y \textbf{smallint}.

    \item \textbf{numeric(p, d)}: p dígitos (más el signo), y d de esos dígitos pertenecen a la parte decimal.

    \item \textbf{real}, \textbf{double precision}.

    \item \textbf{float(n)}: número de coma flotante cuya precisión es, por lo menos, de n dígitos. 
\end{itemize}

Bases de datos:
\begin{itemize}
    \item CREATE DATABASE name [opciones extra]
    
    \item ALTER DATABASE name RENAME TO new\_name
    
    \item DROP DATABASE name
\end{itemize}

Tablas:
\begin{itemize}
    \item CREATE [opciones] TABLE table\_name [opciones, column\_name, data\_type]. (CREATE TABLE Personas(dni INT, nombre VARCHAR(25)))
    
    \item ALTER TABLE name [*] action [...] 

    \item DROP TABLE name [...]
\end{itemize}

\section{Gestión de la información contenida en la BD}
\begin{itemize}
    \item INSERT INTO table\_name [column\_name]. (si el column\_name no se indica va a toda la tabla)

    \item UPDATE table\_name SET [column\_name]

    \item DELETE FROM table\_name [WHERE condition]
\end{itemize}

\newpage

\section{Vistas}
\begin{itemize}
    \item \textbf{Relación base}: una tabla con un nombre correspondiente a una relación del esquema de la BD dónde las tuplas están almacenadas físicamente en el Gestor de BDs.

    \item \textbf{Vista}: el resultado dinámico de una o más operaciones relacionadas sobre las relaciones base (y, opcionalmente, vistas previas) para producir otra relación.
\end{itemize}

\begin{itemize}
    \item CREATE VIEW name [opciones] AS query

    \item ALTER VIEW name col\_name SET DEFAULT expresión

    \item DROP VIEW name

    \item CREATE MATERIALIZED VIEW table\_name AS query: los datos de la vista están almacenados.

    \item REFRESH MATERIALIZED VIEW name: se utiliza para refrescar los datos de la vista.

    \item DROP MATERIALIZED VIEW name
\end{itemize}

\section{Restricciones de Integridad}

Las restricciones de integridad son reglas que aseguran la validez y consistencia de los datos en una base de datos. Se pueden clasificar en dos tipos principales:

\subsection{Tipos de Restricciones}

\begin{itemize}
    \item \textbf{Restricción de un solo atributo:} \texttt{column\_constraint}
    \item \textbf{Restricción de la tabla entera:} \texttt{table\_constraint}
\end{itemize}

\subsection{Tipos de Restricciones Específicas}

Las restricciones que se pueden aplicar incluyen:

\begin{itemize}
    \item \texttt{NOT NULL}: asegura que el atributo no puede tener un valor nulo.
    \item \texttt{CHECK (expresión)}: permite definir una condición que debe cumplirse.
    \item \texttt{DEFAULT default\_expression}: establece un valor por defecto para el atributo.
    \item \texttt{PRIMARY KEY index\_parameters}: define la clave primaria de la tabla.
    \item \texttt{REFERENCES reftable [(refcolumn)]}: establece una relación de referencia con otra tabla.
\end{itemize}

\subsection{Claves Candidatas}

\textbf{Unique} ($A_1, A_2, ...$): Los atributos $A_1, A_2, ...$ forman una clave candidata, lo que significa que ningún par de tuplas en la relación puede ser igual en todos los atributos indicados.

\subsection{Ejemplo de Restricción CHECK}

\begin{itemize}
    \item \texttt{CHECK (sueldo > \space 40000)}: Esta restricción asegura que el sueldo sea mayor a 40,000.
\end{itemize}

\newpage

\subsection{Integridad Referencial}

La integridad referencial asegura que el valor que aparece en una relación para un conjunto dado de atributos también aparezca en otra relación para un conjunto determinado de atributos.

\subsubsection{Ejemplo de Clave Externa}

\begin{itemize}
    \item \texttt{foreign key (nombre\_dept) references departamento}: Esta declaración de clave externa especifica que para cada tupla de la tabla \texttt{asignatura}, el nombre del departamento debe existir en la relación \texttt{departamento}.
\end{itemize}

\subsection{Políticas de Integridad Referencial}

Las políticas de integridad referencial determinan cómo se comporta la base de datos cuando se borran o actualizan datos en la clave candidata referenciada. Las opciones son:

\begin{itemize}
    \item \textbf{NO ACTION}: No se permite la acción (equivalente a RESTRICT).
    \item \textbf{CASCADE}: Al borrar una tupla referenciada, se borran también las tuplas que la referencian.
    \item \textbf{SET NULL}: Permite el borrado de la tupla referenciada y establece a \texttt{NULL} las tuplas que la referencian.
    \item \textbf{SET DEFAULT}: Permite el borrado de la tupla referenciada y establece un valor por defecto en las tuplas que la referencian.
\end{itemize}

\subsection{Orden de Operaciones}

Para las actualizaciones, generalmente se utiliza \texttt{CASCADE}. Para los borrados, el orden correcto es:

\begin{enumerate}
    \item NO ACTION
    \item SET...
    \item CASCADE
\end{enumerate}






\begin{comment}
\begin{figure}[h]
    \centering
    \includegraphics[width=0.5\textwidth]{1.png}
    \caption{}
\end{figure}
\end{comment}

\begin{comment}
\begin{wrapfigure}[]{r}{0.5\linewidth}
    \centering
    \includegraphics[width=\linewidth]{8.png}
    \caption{}
\end{wrapfigure}
\end{comment}

\end{document}