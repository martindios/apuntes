\documentclass{article}

% Language setting
% Replace `english' with e.g. `spanish' to change the document language
\usepackage[spanish]{babel}

% Set page size and margins
% Replace `letterpaper' with `a4paper' for UK/EU standard size
\usepackage[letterpaper,top=2cm,bottom=2cm,left=3cm,right=3cm,marginparwidth=1.75cm]{geometry}

% Useful packages
\usepackage{amsmath}
\usepackage{graphicx}
\usepackage[colorlinks=true, allcolors=blue]{hyperref}

\title{Redes Tema 3. Capa de transporte}
\author{Martín González Dios}

\begin{document}
\maketitle

En el \textbf{origen} prepara los mensajes de las aplicaciones para ser transmitidos y en el \textbf{destino} recupera los mensajes y los entrega a las aplicaciones. Solo está implementada (la capa de transporte) en los sistemas finales. \\

En TCP/IP prepara los mensajes para transmitirlos por un canal no fiable (red de datagramas, IP,
servicio de mejor esfuerzo) usando los protocolos de transporte \textbf{TCP} (fragmenta los mensajes en
segmentos a los que les añade su cabecera) o \textbf{UDP} (solo añade la cabecera). \\

Los mensajes pasan de la capa de aplicación a la de transporte a través de \textbf{sockets}. Los procesos escriben y leen del socket, la capa de transporte recorre todos los sockets abiertos, procesa los mensajes y los envía a la capa de red (\textbf{multiplexación}) y recoge los segmentos de la capa de red para reconstruir los mensajes y colocarlos en su socket destino (\textbf{demultiplexación}), el cual se identifica por los números de puerto de la cabecera del segmento (int 16 bits). \\

En sockets sin conexión (\textbf{UDP}) este se identifica por la IP y el puerto destino, haciendo que segmentos de distinto host con mismo puerto destino sean recogidos por el mismo proceso. \\

\begin{figure}[h]
    \centering
    \includegraphics[width=0.5\textwidth]{Captura_de_pantalla_20241030_165248.png} % Ajusta el ancho
    \caption{Multiplexación con sockets sin conexión (UDP)}
    \label{fig:etiqueta}
\end{figure}

En sockets orientados a conexión (\textbf{TCP}) el socket se identifica por la tupla IP origen, IP destino, puerto origen y puerto destino, por lo que segmentos de distinto host o puerto de origen con igual puerto destino irán a sockets distintos y pueden ser atendidos por procesos/hilos distintos. En TCP hay 1 socket servidor (espera conexiones de clientes) y varios de conexión (transmiten los datos). \\
\begin{figure}[h]
    \centering
    \includegraphics[width=0.5\textwidth]{Captura_de_pantalla_20241030_165949.png} % Ajusta el ancho
    \caption{Multiplexación con sockets orientados a conexión (en TCP)}
    \label{fig:etiqueta}
\end{figure}

\newpage

\section{UDP(Protocolo de datagramas de usuario)}
Es \textbf{simple} y \textbf{poco sofisticado}, hace casi lo mínimo (multiplexación/demultiplexación y comprobación de errores). \\

En origen \textbf{añade} una \textbf{cabecera de 4 campos} (puerto origen, puerto destino, bytes del segmento: cabecera + datos y suma de comprobación) al mensaje \textbf{formando un segmento}. En destino comprueba si el paquete llegó sin errores para entregarlo o no. \\

Es un \textbf{protocolo sin conexión} (sin acuerdo previo emisor-receptor → no hay retardo estableciendo la conexión y sin retransmisión en caso de error) y \textbf{sin estado} (cada segmento es independiente, no hay segmentación ya que no hay información para reconstruir los mensajes, lo que hace más rápido el procesamiento y usa menos recursos). Usa \textbf{cabeceras más pequeñas} y controla más la aplicación. \\
\begin{figure}[h]
    \centering
    \includegraphics[width=0.5\textwidth]{Captura_de_pantalla_20241030_170410.png} % Ajusta el ancho
    \caption{Estructura de UDP}
    \label{fig:etiqueta}
\end{figure}

\section{Transmisión fiable}
Se basa en los protocolos de \textbf{retransmisión de paquetes con errores}. Usan protocolos \textbf{ARQ} (Automatic Repeat reQuest) que pueden \textbf{pasar y esperar} (se envía un paquete y se espera la confirmación) o de \textbf{ventana deslizante} (se envían varios paquetes antes de la confirmación). Se consideran \textbf{enlaces bidireccionales} (full duplex) y se numeran los paquetes de 0 a $2^n$ – 1 (n bits). \\

\newpage

\subsection{Protocolo ARQ parar y esperar} 
\begin{itemize}
    \item Sin errores: receptor devuelve ACK con el no del siguiente paquete.
    \item Pérdida de paquetes/paquete con errores: receptor no devuelve ACK, timeout y retransmisión.
    \item Pérdida ACK: timeout y retransmisión, el receptor recibe un duplicado que descarta y envía ACK.
    \item Timeout: si paquetes y ACKs llegan con retraso se reenvían ignorándolos por duplicados.
\end{itemize}

\begin{figure}[h]
    \centering
    \includegraphics[width=1.1\textwidth]{Captura_de_pantalla_20241030_171145.png} % Ajusta el ancho
    \caption{Casos ARQ parar y esperar}
    \label{fig:etiqueta}
\end{figure}

Hay \textbf{variantes} como \textbf{NAK} que indica la recepción de un paquete con errores y no se espera timeout.  Otra es que 2 ACKs iguales equivalgan a un NAK: un paquete con errores devuelve el ACK del último paquete correcto. O que 3 ACKs equivalgan a un NAK, resolviendo ciertos conflictos. \\
 
El principal \textbf{inconveniente} de este protocolo es la \textbf{poca utilización del enlace}. \\

\subsection{Protocolo ARQ de ventana deslizante}
El emisor envía N paquetes antes de recibir los ACKs. \\
\textbf{Utilización del enlace} por parte del \textbf{emisor}: $U=\frac{t_{trans}}{RTT+t_{trans}}$ \\
\textbf{Tiempo útil} en el emisor: $N*t_{trans}$ \\
\textbf{Tiempo total}: $t_{trans} + RTT$ \\
U = 1 si $N*t_{trans} \geq t_{trans} + RTT \rightarrow N \geq 1 + RTT/t_{trans}$ \\

\begin{figure}[h]
    \centering
    \includegraphics[width=0.4\textwidth]{Captura_de_pantalla_20241030_172116.png} % Ajusta el ancho
    \caption{Utilización del enlace}
    \label{fig:etiqueta}
\end{figure}

\newpage

Se requiere que el \textbf{rango de números de secuencia} abarque al menos el \textbf{doble del tamaño de la ventana emisora} (conjunto de N paquetes que el emisor puede enviar o que están pendientes de confirmación) y que \textbf{emisor y receptor} puedan \textbf{almacenar más de un paquete}. \\

La \textbf{ventana receptora} es el conjunto de N paquetes que el receptor puede aceptar o está procesando.
\begin{itemize}
    \item \textbf{Go Back N} (Retroceder N): el receptor \textbf{sólo acepta paquetes en orden}. Si un paquete llega con errores o no llega se descartan los siguientes y si expira un timeout se retransmiten ese y los siguientes. Los ACKs implican a todos los paquetes previos (acumulativos).
    
    \item \textbf{Repetición selectiva}: el receptor \textbf{acepta paquetes fuera de orden}, retransmitiendo solo los erróneos o los que no llegan. Se debe enviar el ACK de cada paquete recibido.
\end{itemize}

\begin{figure}[h]
    \centering
    \includegraphics[width=0.6\textwidth]{Captura_de_pantalla_20241030_172650.png} % Ajusta el ancho
    \caption{Retroceder N y repetición selectiva}
    \label{fig:etiqueta}
\end{figure}


\section{TCP(Protocolo de control de transmisión)}
Aplica los principios de \textbf{transmisión fiable} y usa números de secuencia (32 bits) que \textbf{identifican bytes, no segmentos}. Empiezan por un número aleatorio x que incrementa en x + n bytes. \textbf{Los ACKs indican el siguiente byte a recibir} y también pueden transmitirse con datos (superposición o piggybacking). Hay un tiempo máximo de espera para mandar datos, si no envía solo el ACK. \\

En este protocolo el emisor usa \textbf{temporizadores para la transmisión}, que se recomienda que sea único (al llegar un ACK o retransmitir un segmento se reinicia). \textbf{Usa ventana deslizante, ACKs acumulativos y solo retransmite segmentos no confirmados}. \\

\begin{figure}[h]
    \centering
    \includegraphics[width=0.2\textwidth]{Captura_de_pantalla_20241030_174737.png} % Ajusta el ancho
    \caption{Números de secuencia}
    \label{fig:etiqueta}
\end{figure}

\newpage

La estimación del tiempo de espera sigue las siguientes \textbf{fórmulas}: \\
\textbf{Temporizador} = EstimacionRTT + 4DevRTT \\
\textbf{EstimacionRTT} = (1-$\alpha$)EstimacionRTT + $\alpha$ MuestraRTT \\
\textbf{DevRTT} = (1-$\beta$)DevRTT + $\beta$ $|$MuestraRTT + EstimacionRTT$|$ \\
DevRTT es una medida de la variación del RTT, y en general $\alpha$=0.125 y $\beta$=0.25  \\
La estimación del RTT es en base a  segmentos transmitidos y no confirmados (no retransmitidos) \\
*Al expirar el temporizador, el emisor duplica el tiempo de espera.\\

\\

Cuando el \textbf{receptor} recibe un segmento con mayor número del esperado \textbf{envía un ACK duplicado} del último segmento correctamente recibido. Al recibir \textbf{3 ACKs duplicados} (el cuarto) se interpreta como un \textbf{NAK} y se realiza una retransmisión rápida. En este caso el problema es \textbf{leve}, \textbf{si expira el temporizador} se pierden los segmentos y los ACKs y se vuelve \textbf{grave}. \\

\begin{figure}[h]
    \centering
    \includegraphics[width=0.35\textwidth]{Captura_de_pantalla_20241030_180145.png} % Ajusta el ancho
    \caption{Retransmisión rápida}
    \label{fig:etiqueta}
\end{figure}

El TCP tiene \textbf{control de flujo} que indica el \textbf{ritmo al que puede recibir datos}. El receptor indica el tamaño de su ventana de recepción, el emisor fija su ventana de envío a ese valor para lo que existe un campo en la cabecera TCP. \textbf{El tamaño de la ventana puede modificarse en cada envío}. \\

\textbf{Conexión}: acuerdo en \textbf{tres fases}:
\begin{itemize}
    \item SYN=1, cuando se envía por primera vez x o y (se consume un número de secuencia)
    \item ACK=1, cuando se confirma un segmento 
    \item En la tercera fase ya se pueden enviar datos
\end{itemize}

\begin{figure}[h]
    \centering
    \includegraphics[width=0.5\textwidth]{Captura_de_pantalla_20241031_093809.png} % Ajusta el ancho
    \caption{Conexión}
    \label{fig:etiqueta}
\end{figure}

\newpage

\textbf{Desconexión}: en \textbf{dos fases}: cada una para desconectar la transmisión en un sentido. Se podría desconectar en un sentido y seguir transmitiendo en el otro. \\
FIN=1, solicitud de desconexión (se consume un número de secuencia)
ACK=1, aceptación

\begin{figure}[h]
    \centering
    \includegraphics[width=0.5\textwidth]{Captura_de_pantalla_20241031_094254.png} % Ajusta el ancho
    \caption{Desconexión}
    \label{fig:etiqueta}
\end{figure}

\\

Una vez establecida la conexión comienza la \textbf{transmisión}. La aplicación pasa los datos a la capa de transporte (escribe en el socket) que los va acumulando. Cuando el número de bytes supera el \textbf{MSS} (maximun segment size), la aplicación fuerza el envío (activa el flag PSH) o un temporizador llega a 0, se dispara la transmisión de un segmento (TPC genera el segmento y se lo pasa a IP). \\
\\
El envío A → B también \textbf{debe incluir el ACK} (piggybacking). 
\begin{itemize}
    \item Si el receptor recibe un segmento en orden, el anterior fue confirmado y no tiene datos para enviar, retrasa el ACK hasta recibir otro segmento o hasta que transcurra cierto tiempo.
    
    \item Si llega un segmento esperado y no se confirmó el anterior se envía el ACK.
    
    \item Si el segmento que llega tiene un no de secuencia mayor del esperado se envía un ACK con el número de secuencia esperado.
    
    \item Si llega un segmento que faltaba se envía un ACK con el siguiente esperado (acumulativo).
    
    \item Si llega un segmento duplicado se descarta y se envía un ACK con el esperado.
\end{itemize}

\begin{figure}[h]
    \centering
    \includegraphics[width=0.5\textwidth]{Captura_de_pantalla_20241031_095046.png} % Ajusta el ancho
    \caption{Cabecera TCP}
    \label{fig:etiqueta}
\end{figure}

\newpage

\section{Control de congestión}

Los recursos de la red deben repartirse entre las \textbf{peticiones}. Si hay demasiados paquetes en la red se producen \textbf{retardos en las transmisiones} y se \textbf{pierden paquetes}. Esto se produce usualmente por el \textbf{desbordamiento de la memoria de los routers}. Ante estas congestiones los elementos de la red realizan esfuerzos como pre-reservar recursos para evitarla o dejar que ocurra y resolverla después. \\
Las \textbf{congestiones} pueden originarse por \textbf{diversas causas}: 
\begin{itemize}
    \item \textbf{2 emisores, 1 router con capacidad infinita y enlace compartido de velocidad C}. Mientras la tasa de entrada $\lambda_{in}$ sea $\leq$ C/2 la tasa de salida $\lambda_{out}$ será igual a la de entrada. \\
    Si $\lambda_{in}$ $>$ C/2 el enlace no puede responder, los paquetes se acumulan en la cola del router y aumenta el retardo.

    \item \textbf{2 emisores, 1 router con memoria finita y enlace compartido de velocidad C}. Siendo $\lambda'_{in}$ la carga ofrecida al enlace con datos originales y retransmitidos, mientras sea $\leq$ C/2 el router puede manejar la carga. Si excede C/2 la memoria del router no llega, causando pérdidas y aumentando el retardo por la retransmisión de paquetes.

    \item \textbf{Varios emisores, routers de memoria finita y varios enlaces}. Mientras la tasa de transmisión es baja los paquetes se envían y procesan correctamente con un retardo finito. Si aumenta demasiado los buffers de los routers se llenan, perdiendo paquetes y disminuyendo la tasa de entrega efectiva $\lambda_{out}$ que puede llegar a 0 por la saturación.
    
\end{itemize}

Por esto se necesitan \textbf{mecanismos de control de congestión}, que en TCP/IP están principalmente en TCP. Estos se basan en la \textbf{adecuación} de la \textbf{tasa de envío} del \textbf{emisor} \textbf{según la congestión} que percibe (considera congestión al expirar un temporizador o recibir 3 ACKs duplicados). \\
En el emisor se definen la \textbf{ventana de congestión} (se actualiza según el caso) y el \textbf{RTT} (se estima periódicamente calculando el tiempo desde el envío de un segmento hasta su ACK). \\
\textbf{Tasa de envío} = $\frac{ventana De Congestión}{RTT}$(bytes/segundo) \\

\\

Para \textbf{actualizar la ventana de congestión} se determina la capacidad inicial de la red con el \textbf{inicio lento} (comienza con una tasa de envío conservadora que va aumentando exponencialmente). En cierto punto se aplica el \textbf{AIMD} (incremento aditivo/decremento multiplicativo) que va aumentando o disminuyendo la ventana para mantener la tasa controlada. En caso de congestión se aplica la \textbf{recuperación rápida}, que reduce la ventana sin llegar al inicio lento. \\

Existe también la \textbf{notificación explícita de congestión} (\textbf{ECN}). Se da cuando un router congestionado activa unos bits en la cabecera IP, haciendo que el receptor TCP active el bit ECE (Eco de ECN) y que el emisor reduzca la ventana de congestión y active el bit CWR. \\

La compartición de un enlace de capacidad limitada no es imparcial.
\begin{itemize}
    \item Dos conexiones TCP irán repartiendo la capacidad del enlace de forma variable.
    \item Entre una conexión TCP y una UDP, la UDP acaparará la mayor parte de la capacidad.
    \item Entre una conexión TCP y 9 conexiones TCP paralelas, la primera usará 1/10 y la segunda 9/10.
\end{itemize}







\end{document}